% !Mode:: "TeX:UTF-8"
%% start of file `template_en.tex'.
%% Copyright 2006-1008 Xavier Danaux (xdanaux@gmail.com).
%
% This work may be distributed and/or modified under the
% conditions of the LaTeX Project Public License version 1.3c,
% available at http://www.latex-project.org/lppl/.


\documentclass[12pt,a4paper,xetex]{moderncv}

%
\usepackage{fontspec,xunicode} % 加载 xetex 的宏包
%xltxtra
% 以下是英文字体的设置
\defaultfontfeatures{Scale=MatchLowercase} % 这个参数保证 serif、sans-serif 和 monospace 字体在小写时大小匹配
\setmainfont[Numbers=OldStyle,Mapping=tex-text]{Times New Roman} % 使用 Garamond 字体,把数字设置为等宽,比较好看
\setsansfont[Mapping=tex-text]{Times New Roman} % 使用 XeTeX 的 text-mapping 方案,正确显示 LaTeX 样式的双引号(`` '')
\setmonofont{Times New Roman}

% 以下是中文字体的设置
\usepackage[slantfont,boldfont,CJKnumber]{xeCJK} % 加载 xeCJK,允许斜体、粗体和 CJK 数字以及 CJK 对空格的设置
%CJKtextspaces
\setCJKmainfont[BoldFont={SimHei}, ItalicFont={KaiTi}]{SimSun}
\setCJKsansfont{KaiTi}
\setCJKmonofont{STFangsong}


\setCJKfamilyfont{song}{SimSun}
\setCJKfamilyfont{kai}{KaiTi}
\setCJKfamilyfont{hei}{SimHei}
\setCJKfamilyfont{hwkai}{STKaiti}
\setCJKfamilyfont{hwfs}{STFangsong}
\setCJKfamilyfont{hws}{STSong}
\setCJKfamilyfont{fs}{FangSong}
\setCJKfamilyfont{yahei}{SimHei}

\newcommand\song{\CJKfamily{song}}
\newcommand\kai{\CJKfamily{kai}}
\newcommand\hei{\CJKfamily{hei}}
\newcommand\hwkai{\CJKfamily{hwkai}}
\newcommand\hwfs{\CJKfamily{hwfs}}
\newcommand\hws{\CJKfamily{hws}}
\newcommand\fs{\CJKfamily{fs}}
\newcommand\yahei{\CJKfamily{yahei}}

\newcommand{\erhao}{\fontsize{22pt}{\baselineskip}\selectfont}
\newcommand{\xiaoerhao}{\fontsize{18pt}{\baselineskip}\selectfont}
\newcommand{\sanhao}{\fontsize{16pt}{\baselineskip}\selectfont}
\newcommand{\xiaosanhao}{\fontsize{15pt}{\baselineskip}\selectfont}
\newcommand{\sihao}{\fontsize{14pt}{\baselineskip}\selectfont}
\newcommand{\xiaosihao}{\fontsize{12pt}{\baselineskip}\selectfont}
\newcommand{\wuhao}{\fontsize{10.5pt}{\baselineskip}\selectfont}
\newcommand{\xiaowuhao}{\fontsize{9pt}{\baselineskip}\selectfont}
\newcommand{\liuhao}{\fontsize{7.5pt}{\baselineskip}\selectfont}



% moderncv themes
%\moderncvtheme[blue]{classic}                 % optional argument are 'blue' (default), 'orange', 'red', 'green', 'grey' and 'roman' (for roman fonts, instead of sans serif fonts)
%\moderncvtheme[green]{classic}                % idem casual
\moderncvtheme[blue,roman]{sld}
% character encoding
%\usepackage[utf8]{inputenc}                   % replace by the encoding you are using

% adjust the page margins
\usepackage[scale=0.8]{geometry}
%\setlength{\hintscolumnwidth}{3cm}						% if you want to change the width of the column with the dates
%\AtBeginDocument{\setlength{\maketitlenamewidth}{6cm}}  % only for the classic theme, if you want to change the width of your name placeholder (to leave more space for your address details
\AtBeginDocument{\recomputelengths}                     % required when changes are made to page layout lengths

% personal data
\firstname{\kai{侍路登}}
\familyname{}
\title{个人简历}               % optional, remove the line if not wanted
\address{浙江省杭州市浙大路38号浙江大学玉泉校区曹光彪西楼404}{邮编:310027}    % optional, remove the line if not wanted
\mobile{18868818796}                    % optional, remove the line if not wanted
%\fax{fax (optional)}                          % optional, remove the line if not wanted
\email{lvtonsmith@yeah.net}
%\email{lvtonsmith@gmail.com}                     % optional, remove the line if not wanted
\extrainfo{http://shiludeng.com} % optional, remove the line if not wanted
\photo[64pt]{sld.jpg}                         % '64pt' is the height the picture must be resized to and 'picture' is the name of the picture file; optional, remove the line if not wanted
\quote{浙江大学~~计算机应用技术专业~~硕士研究生}                 % optional, remove the line if not wanted

%\nopagenumbers{}                             % uncomment to suppress automatic page numbering for CVs longer than one page


%----------------------------------------------------------------------------------
%            content
%----------------------------------------------------------------------------------
\begin{document}
\maketitle

\section{\fs{基本信息}}
\cvcomputer{姓~~~~~~~~名:}{侍路登}{性~~~~~~~~别:}{男}
\cvcomputer{民~~~~~~~~族:}{汉族}{籍~~~~~~~~贯:}{江苏省东海县}
\cvcomputer{出生日期:}{1989-10-05}{政治面貌:}{中共党员}
%%%%%%%%%%%%%%%%%%%%%%%%%%%%%%%%%%%%%%%%%%%%%%%%%%%%%%%%%
\section{\fs{受教育经历}}
\cventry{2004--2007}{\textit{高中}}{江苏省东海高级中学}{江苏省连云港市}{}{}
\cventry{2007--2011}{\textit{本科}}{南京理工大学}{江苏省南京市}{}{}
\cventry{2011--~~~~~~~~}{\textit{硕士}}{浙江大学}{浙江省杭州市}{}{}
\section{\fs{个人能力}}
\cvline{\kai{学习能力}}{研究生入学考试成绩优秀,获得公费名额;研究生学习期间考试成绩优秀;有较强的自学能力;}
\cvline{\kai{理论水平}}{在计算机软件开发、系统分析与设计、数据库设计与管理、计算机网络技术、项目管理、软件工程等相关领域,有坚实的理论基础;}
\cvline{\kai{专业技能}}{ 掌握C语言,C++等程序设计语言,能熟练使用OFFICE办公软件;}
\cvline{\kai{工作能力}}{工作尽职尽责、服从领导,能够认真、按时完成任务,具有良好的团队合作精神;}

\section{\fs{获奖经历}}
\cvlistdoubleitem[\Neutral]{“江苏省首届大学生程序设计竞赛省二等奖”}{“南京理工大学优秀学生党员”;}
\cvlistdoubleitem[\Neutral]{国家奖学金}{}

\section{\fs{项目经历}}

\subsection{本科项目}
\cventry{2009--2010}{开发人员}{}{}{}{}%{Description line 1\newline{}Description line 2}% arguments 3 to 6 are optional

\section{\fs{爱好}}
\cvlistdoubleitem[\Neutral]{羽毛球}{乒乓球}
\cvlistdoubleitem[\Neutral]{中国象棋}{唱歌}


\section{\fs{自我评价}}
\cvline{\kai{性格}}{开朗、活泼、乐观}
\cvline{\kai{学习}}{勤奋严谨,有钻研精神,动手能力强,有很强的自学能力}
\cvline{\kai{生活}}{勤俭节约,吃苦耐劳,热爱集体,乐于助人,积极参加各类公共活动,有良好的人际关系}
\cvline{\kai{工作}}{做事细心,认真负责,能够高效率的完成工作,易于沟通,具有良好的团队意识,热爱自己从事的工作,敢挑重担}

% Publications from a BibTeX file
\newcommand{\refname}{发表论文}
%\nocite{*}
%\bibliographystyle{unsrt}
%\bibliography{shiludeng}       % 'publications' is the name of a BibTeX file
\begin{thebibliography}{99}
% \bibitem {1} Hongtao Huang, Shaobin Huang, Tao Zhang. A Formal Method for Verifying Production Knowledge Base[A]. Internet Computing for Science and Engineering (ICICSE), 2009 Fourth International Conference on[C]. 2009: 19-23.
% \bibitem {2} Tao Zhang, Shaobin Huang, Hongtao Huang. An Operational Semantics for UML RT-Statechart in Model Checking Context[A]. Internet Computing for Science and Engineering (ICICSE), 2009 Fourth International Conference on[C]. 2009: 12-18.

\end{thebibliography}

\end{document}


%% end of file `template_en.tex'.
